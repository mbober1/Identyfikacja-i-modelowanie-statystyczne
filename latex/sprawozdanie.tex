\documentclass[12pt,a4paper]{article}
\topmargin -1.6cm
\addtolength{\textheight}{4cm}
\textwidth  15.5cm

\leftmargin      5mm
\rightmargin     5mm
\oddsidemargin   5mm
\evensidemargin  5mm

\usepackage{hyperref}
\usepackage{polski}
\usepackage[utf8]{inputenc}
\usepackage{graphicx}
\usepackage{units}
\usepackage{sty/style}
\usepackage{float}
\usepackage{mathtools}



\projekt{Modelowanie i identyfikacja}
\autor{Marcin Bober, 249426}
\przedmiot{Identyfikacja i modelowanie statystyczne}
\prowadzacy{Mgr inż. Maciej Filiński}

\begin{document}
\pdfpageheight   297mm
\pdfpagewidth    210mm

\StronaTytulowa
\SpisTresci

\pagebreak

\section{Generator liczb pseudolosowych}
  \subsection{Opis}
  Zadanie polega na implementacji generatora liczb pseudolosowych z rozkładu jednostajnego oraz analizie wyników uzyskanych z jego udziałem. Generator oparty jest na przekształceniu piłokształtnym o równaniu $X_{n+1} = X_n \cdot z - [X_n \cdot z]$ 

  \subsection{Wpływ wartości początkowej X na własności generatora}

  Wartość $Z$ ustawiona została na wartość 51. Wykorzystano 1000 próbek.

  \begin{figure}[H]
    \centering
    \includegraphics[height=0.3\textheight]{figures/Figure_1.png}
    \label{fig:1}
  \end{figure}

  \begin{figure}[H]
    \centering
    \includegraphics[height=0.3\textheight]{figures/Figure_2.png}
    \label{fig:2}
  \end{figure}

  \begin{figure}[H]
    \centering
    \includegraphics[height=0.3\textheight]{figures/Figure_3.png}
    \label{fig:3}
  \end{figure}

  \begin{figure}[H]
    \centering
    \includegraphics[height=0.3\textheight]{figures/Figure_4.png}
    \label{fig:4}
  \end{figure}

  \begin{figure}[H]
    \centering
    \includegraphics[height=0.3\textheight]{figures/Figure_5.png}
    \label{fig:5}
  \end{figure}
  
  \begin{itemize}
    \item Ustawienie wartości początkowej równej zero powoduje że wszystkie wygenerowane próbki są zerowe. (Patrz wykres \ref{fig:1}) Dzieje się tak ponieważ algorytm opiera się o obliczenie iloczynu liczb, których jednym ze składników jest zero.
    \item Wybór liczby całkowitej spowoduje że pierwsza próbka jest równa tej wartości, a wszystkie kolejne są zerowe (Patrz wykres \ref{fig:5}). Wynika to z faktu że obliczana jest reszta z dzielenia wartości przez jeden, która w taki wypadku zawsze równa jest zero.
    \item Zalecanym zakresem wyboru wartości początkowej jest przedział zawierający liczby większe od zera, z pominięciem liczb całkowitych.
  \end{itemize}

  \subsection{Wpływ parametru Z na własności generatora}

  Wartość $X_0$ ustawiona została na wartość 0,01. Wykorzystano 1000 próbek.

  \begin{figure}[H]
    \centering
    \includegraphics[height=0.3\textheight]{figures/Figure_6.png}
    \label{fig:6}
  \end{figure}

  \begin{figure}[H]
    \centering
    \includegraphics[height=0.3\textheight]{figures/Figure_7.png}
    \label{fig:7}
  \end{figure}

  \begin{figure}[H]
    \centering
    \includegraphics[height=0.3\textheight]{figures/Figure_8.png}
    \label{fig:8}
  \end{figure}

  \begin{figure}[H]
    \centering
    \includegraphics[height=0.3\textheight]{figures/Figure_9.png}
    \label{fig:9}
  \end{figure}
  
  \begin{itemize}
    \item Dla zerowego współczynnika $Z$ pierwsza próbka uzyskuje wartość początkowa, a kolejne są zerami. Wynika to z mnożenia tych wyników przez współczynnik $Z$ czyli zero.
    \item Gdy wartość $Z$ jest równa jedności, wszystkie otrzymane wyniki są identyczne z wartością startową.
    \item W przypadku wykorzystania liczb parzystych, uzyskiwane wyniki szybko trafiają na wartość zero, która powoduje zatrzymanie generowania kolejnych wartości losowych. 
    \item Najlepsze wyniki otrzymywane są dla współczynnika $Z$ będącego dużą liczbą pierwszą. 
  \end{itemize}

  \subsection{Okres generatora dla wybranych wartości Z}

  \begin{table}[h!]
    \centering
    \begin{tabular}{ c | c | c }
      $X_0$ & $Z$ & okres generatora  \\ 
      \hline
      0,1 & 1 & 1  \\  
      0,1 & 2 & 4  \\  
      0,1 & 3 & 4  \\  
      0,1 & 4 & 2  \\  
      0,1 & 5 & 1  \\
      0,1 & 6 & 1  \\
      0,1 & 7 & 4  \\
      0,1 & 8 & 4 
    \end{tabular}
    \caption{Okres generatora w zależności od wartości Z}
    \label{table:1}
  \end{table}

  \subsection{Podobieństwo histogramu ciągu wygenerowaych liczb, a gęstość rozkładu jednostajnego}


  \begin{figure}[H]
    \centering
    \includegraphics[height=0.25\textheight]{figures/Figure_10.png}
    \caption{Ilość wygenerowanych próbek - 20}
    \label{fig:10}
  \end{figure}

  \begin{figure}[H]
    \centering
    \includegraphics[height=0.25\textheight]{figures/Figure_11.png}
    \caption{Ilość wygenerowanych próbek - 200}
    \label{fig:11}
  \end{figure}

  \begin{figure}[H]
    \centering
    \includegraphics[height=0.25\textheight]{figures/Figure_12.png}
    \caption{Ilość wygenerowanych próbek - 2000}
    \label{fig:12}
  \end{figure}

  \begin{figure}[H]
    \centering
    \includegraphics[height=0.25\textheight]{figures/Figure_13.png}
    \caption{Ilość wygenerowanych próbek - 20000}
    \label{fig:13}
  \end{figure}

  \begin{itemize}
    \item Im więcej wygenerowaych próbek tym mocniej histogram upodabnia się do gęstości prawdopodobieństwa rozkładu jednostajnego.  
  \end{itemize}

\section{Generator dany równaniem}

\subsection{Opis}
Zadanie polega na implementacji generatora liczb pseudolosowych oraz analizie wyników uzyskanych z jego udziałem. Generator oparty jest na równaniu: 

$X_{n+1}=(a_0X_{n} + a_1X_{n-1} + \ldots + a_{k}X_{n-k} + C)$mod $m$


\subsection{Zależność od współczynnika m}
Współczynnik $m$ odpowiada za zakres generowanych wartości.

\begin{figure}[H]
  \centering
  \includegraphics[width=1\textwidth]{figures/Figure_14.png}
  \caption{Zakres generowania liczb [0, 0.0174]}
  \label{fig:14}
\end{figure}

\begin{figure}[H]
  \centering
  \includegraphics[width=1\textwidth]{figures/Figure_15.png}
  \caption{Zakres generowania liczb [0, 1]}
  \label{fig:15}
\end{figure}


\subsection{Zależność od współczynnika k}
Współczynnik $k$ odpowiada za ilość poprzednich próbek używanych podczas generowania nowych wartości. 

\begin{figure}[H]
  \centering
  \includegraphics[width=0.8\textwidth]{figures/Figure_20.png}
  \caption{Współczynnik k=1}
  \label{fig:14}
\end{figure}

\begin{figure}[H]
  \centering
  \includegraphics[width=0.8\textwidth]{figures/Figure_21.png}
  \caption{Współczynnik k=2}
  \label{fig:14}
\end{figure}

\begin{figure}[H]
  \centering
  \includegraphics[width=0.8\textwidth]{figures/Figure_22.png}
  \caption{Współczynnik k=3}
  \label{fig:14}
\end{figure}


\begin{itemize}
  \item Współczynnik $m$ odpowiada za zakres generowanych wartości. Wybór liczby całkowitej może spowodować zatrzymanie generatora.
  \item Ustawienie współczynnika $k$ na wartość równą jeden uniemożliwia generowanie wartości losowych.
  \item Im większa wartość współczynnika $k$ tym lepsze wyniki generatora.
\end{itemize}

\section{Metoda odwracania dystrybuanty}
  \subsection{Opis}
    Metoda odwracania dystrybuanty polega na odwróceniu funkcji dystrybuanty. Do uzyskania funkcji dystrybuanty posłużymy się całkowaniem funkcji opisującej gęstość prawdopodobieństwa analizowanego rozkładu.

  \subsection{Rozkład numer 1}
  
    Równanie funkcji rozkładu gęstość prawdopodobieństwa:
    \begin{equation}
      f(x) = \begin{cases}
          2x \quad dla \quad x  \in [0,1]\\
          0 \quad dla \quad x  \in (-\infty, 0) \cup (1, \infty)\\
        \end{cases}   
    \end{equation}

    Dystrybuanta:
    \begin{equation}
      F(x) = \begin{cases}
          0 \quad dla \quad x  \in (-\infty, 0)\\
          x^2 \quad dla \quad x  \in [0, 1]\\
          1 \quad dla \quad x  \in (1, \infty)\\
        \end{cases}   
    \end{equation}

    Odwrotna dystrybuanta:
    \begin{equation}
      F^{-1}(y) = \sqrt{y}.
    \end{equation}

  \begin{figure}[H]
    \centering
    \includegraphics[width=1\textwidth]{figures/Figure_16.png}
    \label{fig:16}
  \end{figure}


  \subsection{Rozkład numer 2}
  
  Równanie funkcji rozkładu gęstość prawdopodobieństwa:
  \begin{equation}
    f(x) = \begin{cases}
        x + 1 \quad dla \quad x  \in (-1, 0)\\
        -x + 1 \quad dla \quad x  \in [0, 1)\\
        0 \quad dla \quad x  \notin (-1, 1)\\
      \end{cases}   
  \end{equation}

  Dystrybuanta:
  \begin{equation}
    F(x) = \begin{cases}
        \frac{1}{2} - \frac{x^2}{2} + x \quad dla \quad x  \in [0, 1)\\
        \frac{1}{2} + \frac{x^2}{2} + x \quad dla \quad x  \in (-1, 0)\\
        0 \quad dla \quad x  \in (-\infty, -1]\\
        1 \quad dla \quad x  \in [1, \infty)\\
      \end{cases}   
  \end{equation}

  Odwrotna dystrybuanta:
  \begin{equation}
    F^{-1}(y) = \begin{cases}
          \sqrt{2y} -1 \quad dla \quad x  \in [0, \frac{1}{2}]\\
          1 -\sqrt{2 - 2y} \quad dla \quad x  \in (\frac{1}{2}, 1]\\
        \end{cases}   
  \end{equation}

  \begin{figure}[H]
    \centering
    \includegraphics[width=1\textwidth]{figures/Figure_17.png}
    \label{fig:17}
  \end{figure}

  \subsection{Rozkład wykładniczy}
  
  Równanie funkcji rozkładu gęstość prawdopodobieństwa:
  \begin{equation}
    f(x) = e^{-x}  \quad dla \quad x  \in [0, \infty)
  \end{equation}

  Dystrybuanta:
  \begin{equation}
    F(x) = 1 - e^{-x}  \quad dla \quad x  \in [0, \infty)
  \end{equation}

  Odwrotna dystrybuanta:
  \begin{equation}
    F^{-1}(y) = -ln(1 - y)  \quad dla \quad x  \in [0, \infty)
  \end{equation}


  \begin{figure}[H]
    \centering
    \includegraphics[width=1\textwidth]{figures/Figure_18.png}
    \label{fig:18}
  \end{figure}

  \subsection{Rozkład Laplace'a}
  
  Równanie funkcji rozkładu gęstość prawdopodobieństwa:
  \begin{equation}
    F(x) = \frac{1}{2} e^{-|x|} 
  \end{equation}

  Dystrybuanta:
  \begin{equation}
      F^{-1}(y) = \begin{cases}
            \frac{1}{2} + \frac{1}{2} (1 - e^{-x})  \quad dla \quad x  \in [0, \infty)\\
            \frac{1}{2} - \frac{1}{2} (1 - e^{-x})  \quad dla \quad x  \in (-\infty, 0)\\
          \end{cases}   
  \end{equation}

  Odwrotna dystrybuanta:
    \begin{equation}
      F^{-1}(y) = \begin{cases}
            -ln(1 - y) \quad dla \quad x  \in [0, \infty)\\
            ln(1 + y) \quad dla \quad x  \in (-\infty, 0)\\
          \end{cases}   
    \end{equation}

  \begin{figure}[H]
    \centering
    \includegraphics[width=1\textwidth]{figures/Figure_19.png}
    \label{fig:19}
  \end{figure}


\section{Podsumowanie}

\end{document}